\chapter{Testing}
\newpage

\begin{table}[h!]
\begin{center}
\begin{tabular}{ |p{1cm}|p{3cm}|p{4cm}|p{4cm}|p{2cm}| }
 \hline
 \multicolumn{5}{|c|}{Docstash Test Cases} \\
 \hline
Sr. No.& Test Case Description &Expected Result &Actual Result &Remark\\
\hline
 1 & Google and Facebook Integration and Login &	The user is prompted to select a Google account to sign in with. If you requested scopes beyond profile, email, and openid, the user is also prompted to grant access to the requested resources. &	Get users into your apps quickly and securely, using a registration system they already use and trust—their Google account. &	PASSED\\
 \hline
 2 & Link shortner  & Link should be shortened and if user visits the link he should be redirected to the original link.&	User gets the working shortened link.&	PASSED\\
 \hline
 3 &File upload and retrieval	& Server should accept various kinds of file being uploaded &	User is able to upload file and gets relevant information.&	PASSED\\
 \hline
 4 & File encryption &File should be encrypted before storing to storage. & File is getting encrypted with AES-256 algorithm and saved to the storage. &	PASSED\\
\hline

\end{tabular}
\caption{Doctash Test Cases}
\end{center}
\end{table}

\newpage

\begin{figure}[h]
\begin{center}
    \psfig{figure=./img/test1.png,height=2.3in}
    \caption{Test Execution}
    \label{1}
\end{center}
\end{figure}

\begin{figure}[h]
\begin{center}
    \psfig{figure=./img/test2.png,height=2.3in}
    \caption{Form Validation Test}
    \label{1}
\end{center}
\end{figure}

\newpage
The various testing methodologies implemented are, 

\section{Unit And Integration Testing}
The various modules of the application (Eg- Events module, Contacts module etc) were tested separately and any bugs found were fixed. After successfully finishing unit testing, all items were slowly integrated one by one wile continuously testing for compatibility testing. Finally, after all the modules were integrated, the system was again tested as a whole for proper operation.

\section{White Box Testing}
Some of the white box testing techniques used are,

\begin{itemize}
  \item [1.] Control flow analysis. The various branches of control flow were traversed atleast once.
  \item [2.] The logical decisions were evaluated.
  \item [3.] All the loops were evaluated to check their correct operation.
\end{itemize}

\section{Black Box Testing}
Some of the Black box testing techniques used are,

\begin{itemize}
  \item [1.] The menu ordering functionality was checked using boundary value analysis for all, none and few items ordered.
  \item [2.] The outputs of the various tabs were checked for correctness.
  \item [3.] Login and register functionalities were tested for correct operation.
  \item [4.] In the register functionality, the bounds on the length of username and passwords were tested.
  \item [5.] The mechanism for adding and editing items in the web application was tested for each area of function.

\end{itemize}

\section{Usability Testing}
The user interface of the application as well as the web app were tested for ease of use. Ambigous and unsuitable elements were replaced. It was verified that the application was simple and easy to use for both the residents and the administrators.

\section{Alpha and Beta Testing}
The user interface of the application as well as the web app were tested for ease of use. Ambiguous and unsuitable elements were replaced. End users were allowed to operate the application both under supervision and unsupervised. Their opinions were taken into consideration and required changes made. 
